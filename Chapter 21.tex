\documentclass[aspectratio=169]{beamer}
\usetheme{Boadilla}
\usepackage{graphicx}
\usepackage{amsmath}

    \title{Elements of Microeconomics}
    \author{Hank Behaeghel}
    \institute{Johns Hopkins University}
    \date{Chapter 21}

\begin{document}

\maketitle

\begin{frame}{Where Are We Going?}
    \begin{itemize}
        \item So far, we have discussed market equilibrium, various policiy implications, and we have explored consumer behavior
        a little with the concept of elasticity.
        \item Now we will be taking our studies of consumer behavior even further.
        \item Enter: \textbf{The Theory of Consumer Choice}
    \end{itemize}
\end{frame}

\begin{frame}{Key Concepts That Will Guide Us}
    \begin{itemize}
        \item This is where we clearly introduce some optimization into economics. What do I mean by this?
        \item Think about yourself for a moment. Everytime you want to buy something you are faced with a myriad of questions.
        \begin{itemize}
            \item <2->How much do I want this thing?
            \item <3->How much am I willing to spend? Can I even afford it?
            \item <4->Will it actually be worth having?
            \item <4->etc. etc.
        \end{itemize}
       \item  <5-> All of these questions really boil down to two main questions:
        \begin{enumerate}
            \item <6-> \textbf{Can I afford it?}
            \item<7-> \textbf{Do I actually want it?}
        \end{enumerate}
    \end{itemize}
\end{frame}

\begin{frame}{Can I Afford it?}   
\begin{itemize}
    \item Before you buy something you have to be able to afford it. The amount of disposable income depends on a host of factors.
    \item For the purposes of this class we will mainly focus on the trade off of two goods.
    \item We can model what we can affor using what is known as a \textbf{budget constraint}.
\end{itemize}
\end{frame}

\begin{frame}{The Budget Constraint}
    \begin{itemize}
        \item \textbf{Definition}: the limit on the consumption bundles that a consumer can afford.
        \item \textit{So what does this actually mean?}
        \begin{itemize}
            \item You cannot buy more things than you can afford. This is an obvious point but, its helpful for maximizing our utility.
            \item I can have a certain amount of X and a certain amount of Y.
        \end{itemize}
    \end{itemize}
   
    \begin{center}
        \includegraphics[width=0.5\textwidth, height=0.6\textheight]{/Users/hank/Desktop/BC.png}
    \end{center}
\end{frame}

\begin{frame}{What Do I Want?}
    \begin{itemize}
        \item In economics we use the term "utility" as a way to describe the benefit we gain from an item. You will study this more in terms to come.
        \item The want/value is encoded in this utiltiy
        \item Utility in this class is represented by our \textbf{indifference curve}.
        \item This curve is very important and has some fundemental properties.
    \end{itemize}
\end{frame}

\begin{frame}{The Indifference Curve}
    \begin{itemize}
        \item<1-> The 4 Properties of Indifference Curves (IC):
        \begin{enumerate}
            \item<2-> Higher IC are preferred to lower ones.
            \item<3-> IC's are downward sloping (think about why this is)
            \item<4-> IC \textbf{do not} cross.
            \item<5-> IC are bowed inwards. 
        \end{enumerate}
    \item<6-> The slope of the IC is the \textbf{marginal rate of substitution}.
    \begin{itemize}
        \item \textbf{Definition}: the rate at which a consumer is willing to trade one good for another.
    \end{itemize}
    \end{itemize}
\end{frame}

\begin{frame}{Practice Problems}
\begin{block}{Question 1}
    Carla buys Xbox (X) for \$10 each and Yoyo (Y) for \$4 each. She has an income of \$75.  How many units of good X can Carla afford? Can Carla afford to buy 4 units of good X and 10 units of good Y? If Carla's income increases by 10\%, what happens to the slope of the budget constraint? What is Carla's MRSXY at the utility-maximizing point?
\end{block}
\end{frame}


\end{document}