\documentclass[aspectratio=169]{beamer}
\usetheme{Boadilla}
\usepackage{graphicx}
\usepackage{amsmath}
\usepackage{tabularx}
\usepackage{biblatex}
\usepackage{enumitem}
\usepackage{csquotes}
\usepackage{lmodern}

    \title{Elements of Microeconomics}
    \author{Hank Behaeghel}
    \institute{Johns Hopkins University}
    \date{Chapter 2}

\addbibresource{microslides.bib}

\begin{document}

\maketitle

\begin{frame}{What is an Economist?}
    Mankiw describes an economist as being two things simultaneously:
    \vspace{10pt}
    \begin{itemize}[label = $\bullet$, itemsep = 10pt]
        \item \textbf{Scientists}: we as economists use experiments much like other empirical sciences in order to derive conclusions about the world we inhabit.
        \item \textbf{Policy Advisors}: here the role of an economist is to not just observe the world but improve it through recommendations based on evidence. 
    \end{itemize}
    \vspace{10pt}
    Now this is a fine definition but, I'd like us to consider this next definition that influenced Mankiw's definition (and the name of his dog!). We will turn to John Maynard Keynes.
\end{frame}

\begin{frame}{To the Eloquent}

\begin{quote}
    \enquote{the master-economist must possess a rare combination of gifts. 
    He must reach a high standard in several
    different directions and must combine talents not often found
    together. He must be mathematician, historian, statesman,
    philosopher-in some degree. He must understand symbols
    and speak in words. He must contemplate the particular in
    terms of the general, and touch abstract and concrete in the same
    flight of thought. He must study the present in the light of
    the past for the purposes of the future. No part of man's nature
    or his institutions must lie entirely outside his regard. He must
    be purposeful and disinterested in a simultaneous mood; as
    aloof and incorruptible as an artist, yet sometimes as near the earth
    as a politician. Much, but not all, of this ideal many-sidedness
    Marshall possessed. But chiefly his mixed training and divided
    nature furnished him with the most essential and fundamental
    of the economist's necessary gifts-he was conspicuously historian
    and mathematician, a dealer in the particular and the general,
    the temporal and the eternal, at the same time.}\footfullcite{Master-Economist}
\end{quote}

\end{frame}

\begin{frame}{Important Takeaways}
 \begin{itemize}[label = $\bullet$, itemsep = 10pt]
    \item Notice where Mankiw drew inspiration from Keynes to build his definition.
    \item Those who have read the chapter may be drawn to Keynes' line about being both purposeful and disinterested.
    \begin{itemize}
        \item [(i)] This relates back to the idea of \textit{positive} v. \textit{normative} statements.
        \item [(ii)] We as economists need to be aware of the task at hand. Describing how things are is \textit{positive}, how things ought to be is a \textit{normative} discussion.
    \end{itemize}
 \end{itemize}
    
\end{frame}

\begin{frame}{Our First Model}
    \begin{itemize}[label = $\bullet$, itemsep = 10pt]
        \item Yes circular flow is the first model in the book however, the first model we will study with rigor is known as the
        \textbf{production possibilities frontier}. Sometimes referred to as the PPF or PPC.
        \item The PPF is helpful because it allows us to understand a few important concepts: \textit{opportunity cost, gains from trade, specialization.}
    \end{itemize}
\end{frame}

\begin{frame}{The PPF}

\begin{center}
    \includegraphics[width = 0.8\textwidth, height = 0.9\textheight]{/Users/hank/Desktop/PPF.png}
\end{center}
\end{frame}


\end{document}