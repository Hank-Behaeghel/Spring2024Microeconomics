\documentclass[aspectratio=169]{beamer}
\usetheme{Boadilla}
\usepackage{graphicx}
\usepackage{amsmath}
\usepackage{tabularx}
\usepackage{biblatex}

    \title{Elements of Microeconomics}
    \author{Hank Behaeghel}
    \institute{Johns Hopkins University}
    \date{Chapter 1}

\addbibresource{microslides.bib}

\begin{document}

\maketitle

\begin{frame}{Introduction}
    \begin{itemize}
        \item My name is Hank and I am one of the Junior Lecturers in the Economics Department.
        \item email: \textbf{hbehaeg1@jhu.edu}
        \item Office: \textbf{Wyman 535}
        \item OH: TBD, I will be sending out a poll.
        \item Section: Krieger 205, 9AM on Thursdays.
    \end{itemize}
\end{frame}

\begin{frame}{What's this Economcis Thing?}
    \begin{itemize}
        \item I imagine most of you took macroeconoimcs last semester, if you did not that that will not affect you in this course. 
        \item The book defines economics as \textit{the study of how society manages its scarce resources}. This is a fine definition however note that Mankiw elaborates
        his definition later in the same passage.
        \item Economics has many subfields beyond simply the macro and micro distinction.
        \item Today we will describe some of the guiding principles of the science, as well as what it is economists actually do.
    \end{itemize}

\end{frame}

\begin{frame}{The Guiding Framework}
    \begin{itemize}
        \item Before we dive into models and anything of the sorts we must first build a little intution.
        \item This intution will start with the \textit{10 Principles of Economics}. 
        \item It is important to take these as a framework upon which we will build out models and interpret various ongoings in the course.
        \item Do not spend time memorizing which number principle is which. Rather focus on learning and internalizing how they might shape our thinking regarding economic issues.
    \end{itemize}
\end{frame}

\begin{frame}{How People Make Decisions}
    \begin{itemize}
        \item The first group of principles concerns itself with how people make decisions for themselves.
        \begin{enumerate}
            \item \textbf{People face tradeoffs}. The largest tradeoff that we as a society face is \textit{equality} v. \textit{efficiency}.
            \item \textbf{The cost of something is what you give up to get it}. In economcis we introduce the concept of \textit{opportunity cost}. Opportunity cost is a whollistic way to define costs, it includes things such as: forgone opportunity, time, etc.
            \item \textbf{Rational people think at the margin}. Here the important takeaway is the \textit{margin}. Marginal benefits and costs come up time and time again. 
            \item \textbf{People respond to incentives}. Think bonuses given for acheivement and work well done. Given an incentive poeple tend to seek it out.
        \end{enumerate}
    \end{itemize}
\end{frame}

\begin{frame}{How People Interact}
    \begin{itemize}
        \item The next set of principles sheds some light on how people may interact.
        \begin{enumerate}
            \item \textbf{Trade can make everyone better off}. Here pay attention to the fact that when it says \textit{everyone} it is referring to the aggregate surplus.
            \item \textbf{Markets are usually a good way to organize economic activity}. This principle has a little Hobbes/Locke flavor to it. 
            \item \textbf{Governments can sometimes improve market outcomes}. This prinicple is important because it introduces another actor into the economy. We will dive deeper into this principle's implication later in the course.
        \end{enumerate}
    \end{itemize}
\end{frame}

\end{document}
