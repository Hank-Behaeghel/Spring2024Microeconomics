\documentclass[aspectratio=169]{beamer}
\usetheme{Boadilla}
\usepackage{graphicx}
\usepackage{amsmath}
\usepackage{tabularx}
\usepackage{biblatex}
\usepackage{csquotes}
\usepackage{lmodern}
\addtolength{\itemsep}{15pt}

    \title{Elements of Microeconomics}
    \author{Hank Behaeghel}
    \institute{Johns Hopkins University}
    \date{Chapter 3}

\addbibresource{microslides.bib}

\begin{document}

\maketitle

\begin{frame}{Outline}
    \begin{itemize}
        \item Today we will be diving deeper in our discussion of trade.
        But, this is a micro class why one earth are we talking about trade?
        \item Remember that what we are studying is how people interact within markets. Trade happens between individuals all the time!
        \item The main definitions/concepts we want to take away from today:
        \begin{enumerate}
            \item Be able to define \textbf{comparative and absolute advantage} and identify \textit{who} has it.
            \item Identify a trade price that all parties would accept.
            \item Solve questions concerning trade, we will do a couple today.
        \end{enumerate}
    \end{itemize}
\end{frame}

\begin{frame}{Advantages}
What do we as economists mean by advatages?
    \begin{center}<2->
    \includegraphics[width = 0.5\textwidth, height = 0.7\textheight]{/Users/hank/Desktop/ObiWan.png}
    \end{center}
\end{frame}


\begin{frame}{Absolute Advantage}
    This is Obi-Wan having the high ground.
   \begin{block} <1->{Absolute Advantage}
        \textbf{Absolute advantage (AA)} is the ability to produce a good using fewer inputs than another producer.
    \end{block}
    
    \begin{itemize}
        \item<2-> The key part is this definition is the \textit{fewer inputs} portion.
        \item<3-> What this means is that the party with the absolute advatage is \textbf{more efficient} at producing a good than the other party.
    \end{itemize}
\end{frame}

\begin{frame}{Comparative Advantage}
    Remember opportunity cost from before? It \textbf{never} goes away. Here the advantage between producers is define in terms of opportunity cost. Remember
    that \textbf{OC} is \textit{whatever must be given up to obtain some item.}

    \begin{block}<2->{Comparative Advantage}
        \textbf{Comparative advantage (CA)} the ability to produce a good at a lower opportunity cost than another producer.
    \end{block}

    \begin{itemize}
        \item<3-> Notice the change in the terms used to measure advantage. Now advatange is determined by who sacrifices less to produce.
        \item <4-> One can have a comparative advantage without having the absolute advatage. (This is a \textbf{great} T/F question.)
    \end{itemize}
\end{frame}

\begin{frame}{Trade Price}
    We have two of the things we need to solve most questions we ask you about trade. The last dimension we need is the actual \textit{price} that two parties will trade at.

    \begin{block}<2->{Trade Price}
        The price of trade between two parties must lie between the two opportunity cost. An important consequence of this is that the trade price can take \textbf{any value} in the interval between the opportunity costs. 
    \end{block}
    
    A small note is needed to address units. When discussing trade in this class we will often give you two good, the opportunity cost for each person is in terms of the \textit{other} good. Therefore,
    the trade price MUST be in terms of the goods discussed.
\end{frame}

\begin{frame}{Practice Problem \#1}
    \begin{block}{Cars and Grain}
        American and Japanese workers can each produce 4 cars a year. An American worker can produce 10 tons of grain a year, whereas a Japanese worker can produce 5 tons of grain a year. To keep things simple, assume that each country has 100 million workers.
        \begin{enumerate}
            \item Solve for the total output of each country and graph their respective PPFs.
            \item For the United States, what is the opportunity cost of a car? Of grain? For Japan, what is the opportunity cost of a car? Of grain?
            \item Identify the country with the CA for each good and AA for each good.
            \item Without trade, half of each country’s workers produce cars and half produce grain. What quantities of cars and grain does each country produce? Plot this point on each PPF.
            \item What is a trade price that both countries would accept?
        \end{enumerate}
    \end{block}
\end{frame}

\begin{frame}{Practice Problem \#2}
    \begin{block}{Animal House}
        Pat and Kris are roommates. They spend most of their time studying (of course), but they leave some time for their favorite activities: making pizza and brewing root beer. Pat takes 4 hours to brew a gallon of root beer and 2 hours to make a pizza. Kris takes 6 hours to brew a gallon of root beer and 4 hours to make a pizza. 
        \begin{enumerate}
            \item What is each roommate’s opportunity cost of making a pizza? Who has the absolute advantage in making pizza? Who has the comparative advantage in making pizza?
            \item The price of pizza can be expressed in terms of gallons of root beer. What is the highest price at which pizza can be traded that would make both roommates better off? What is the lowest price? Explain. 
        \end{enumerate}
    \end{block}
\end{frame}
\end{document}