\documentclass[aspectratio=169]{beamer}
\usetheme{Boadilla}
\usepackage{graphicx}
\usepackage{amsmath}

    \title{Elements of Microeconomics}
    \author{Hank Behaeghel}
    \institute{Johns Hopkins University}
    \date{Chapter 21}

\begin{document}

\maketitle

\begin{frame}{Where Are We Going?}
    \begin{itemize}
        \item Last week we reviewed the costs of production. These vary depending on the structure of the market.
        \item This week we will be taking a look at our first market structure.
        \item Enter: \textbf{perfectly competitive markets}
        \item Also your exam is next week and will cover chapters: 6, 7, 8, 21, 13, 14.
    \end{itemize}
\end{frame}

\begin{frame}{Actors and the Basics}
    \begin{itemize}
        \item Throughout this course we have talked about consumers and suppliers. When discussing the perfectly competitive in this chapter the \textit{firms} will be the suppliers.
        \vspace*{5mm}
        \item Criteria to be a perfectly competitive market:
        \vspace*{5mm}
        \begin{enumerate}
            \item<2-> There are \textbf{many} buyers \textit{and} sellers.
            \begin{itemize}
                \item<3-> The reason that this helps characterize a perfectly competitive market is that, with tons of buyers and sellers no one buyer or seller has too much market power. Here buyers and sellers are referred to as \textbf{price takers}.
                \item<4-> Think of you and I in the market for boxed mac and cheese. No company will notice if our section boycotts their boxed mac and cheese. 
            \end{itemize}
            \item<5-> The goods in the market are homogeneous. Essentially there are no meaningful differences between gooods.
            \item<6-> There is free entry and exit.
        \end{enumerate}
    \end{itemize}
\end{frame}

\begin{frame}{The Relationship Between P and MR}
    \begin{itemize}
        \item The fact that firms are price takers in perfectly competitive markets will guide the intuition for the following idnetity.
        \begin{block}<2->{P and MR}
            In \textbf{perfectly competitive} markets. A firm's marginal revenue is equal to the market price. \\
            \vspace{2mm}
            \begin{center}
                $P = MR$\\
            \end{center}
        \end{block}
        \item This arguably the most important idnetity for solving the computational parts of perfectly competitive market questions. 
    \end{itemize}
\end{frame}

\begin{frame}{Making Money}
    \begin{itemize}
        \item Firms operate to \textit{maximize profit}. Whether they are able to or not depends on many things: costs, prices, governemnt regulation.
        \item For our purposes there are a few things that help us analyze whether a firm is maximizing its profit or not.
        \item<2-> Is MC above or below MR?
        \begin{itemize}
            \item<3-> If MC is below MR the firm is leaving money on the table.
            \item<4-> Once MC is above MR then the firm loses money for each additional unit it produces.
        \end{itemize}
    \end{itemize}
\end{frame}

\begin{frame}{The Question Within Ourselves}
    \begin{itemize}
        \item Built into the goal of profit maximization is the decsion to produce.
        \vspace{2mm}
        \item This decsion can be broken down into a few steps:
        \vspace{2mm}
        \begin{enumerate}
            \vspace{2mm}
            \item<2-> To produce or not to produce?
            \begin{itemize}
                \item<3-> If yes move to step 2, if no go on vacaton.
            \end{itemize}
            \item<4-> How much do I produce?
        \end{enumerate}
        \vspace{2mm}
        \item<5-> Both of these questions require some analysis on our part.
    \end{itemize}
\end{frame}

\begin{frame}{When to Produce}
    \begin{itemize}
        \item Everyone's favorite microeconomics buzzwords are back: \textbf{time horizons}.
        \item How we go about our analysis of whether or not to produce (Question 1 from the prior slide) will depend on which time horizon we are considering.
        \item In the \textbf{short run} our production decision revolves around the \textbf{AVC}.
        \item In the \textbf{long run} our production decision revolves around the \textbf{ATC}.
    \end{itemize}
\end{frame}

\end{document}